\documentclass[aspectratio=1610]{beamer}
\usepackage[latin1]{inputenc}
\usepackage[spanish]{babel}
\usepackage{graphicx}
\graphicspath{{images/}}
\usepackage{multicol}
\usepackage{enumitem}
\usepackage{caption}
\usepackage{longtable}
\usepackage{array} % For tables with fixed width
\usepackage[sharp]{easylist}
\usepackage{amsmath}

\usepackage{listings}
\usepackage{tcolorbox}
\tcbuselibrary{most}

\usepackage{csquotes}
\usepackage[style=apa, backend=biber]{biblatex}
\addbibresource{main.bib}

\usefonttheme[onlymath]{serif}
\usetheme{boxes}
\title{Disertaci�n: Uso de \LaTeX\ en la creaci�n de documentos acad�micos}
\subtitle{Innovaci�n Docente e Iniciaci�n a la Investigaci�n Educativa. Especializaci�n en Ingenier�a y Ense�anzas T�cnicas}
\date{Marzo de 2025}

% Indice de secci�n antes de cada secci�n
\AtBeginSection[]
{
  \begin{frame}[shrink]{}  
    \tableofcontents[currentsection,sectionstyle=show/shaded,subsectionstyle=show/show/hide]
  \end{frame}
}

% T�tulos autom�ticos para cada frame, usando el t�tulo de secci�n y subsecci�n
\addtobeamertemplate{frametitle}{
   \let\insertframetitle\insertsectionhead}{}
\addtobeamertemplate{frametitle}{
   \let\insertframesubtitle\insertsubsectionhead}{}

\makeatletter
  \CheckCommand*\beamer@checkframetitle{\@ifnextchar\bgroup\beamer@inlineframetitle{}}
  \renewcommand*\beamer@checkframetitle{\global\let\beamer@frametitle\relax\@ifnextchar\bgroup\beamer@inlineframetitle{}}
\makeatother

\newcommand{\ejemplo}[2]{
  \begin{tcolorbox}[title=#1]
    \lstinputlisting[language=tex]{#2}
    \tcblower%
    \input{#2}
  \end{tcolorbox}
}

\newcommand{\ejemplopdf}[3]{
  \begin{tcolorbox}[title=#1]
    \lstinputlisting[language=tex]{#2}
    \tcblower%
    \includegraphics[width=\textwidth]{#3}
  \end{tcolorbox}
}

\newcommand{\insideFrame}[1]{\begin{frame}[shrink]#1\end{frame}}

\setbeamercovered{transparent}

\begin{document}

\frame{\titlepage}

\section{�Qu� es \LaTeX\ ?}
\begin{frame}
  Seg�n \textcite{LaTeXWik47:online}, \LaTeX\ es:
  \begin{itemize}
    \item Sistema de composici�n de textos orientado a la creaci�n de documentos escritos que presenten una alta calidad tipogr�fica.
    \item Por sus caracter�sticas y posibilidades, se usa de forma especialmente intensa en:
          \begin{itemize}
            \item Generaci�n de art�culos.
            \item Libros cient�ficos.
            \item Expresiones matem�ticas.
          \end{itemize}
  \end{itemize}
\end{frame}

\section{Estructuras b�sicas}

\insideFrame{\ejemplopdf{Documento b�sico}{ejemplos/documento.tex}{ejemplos/documento.png}}
\insideFrame{\ejemplo{Lista de items}{ejemplos/list_itemize.tex}}
\insideFrame{\ejemplo{Lista numerada}{ejemplos/list_enumerate.tex}}
\insideFrame{\ejemplo{Lista numerada de otra manera}{ejemplos/list_easylist.tex}}

\section{Estructuras avanzadas}
\insideFrame{\ejemplo{Ecuaciones matem�ticas}{ejemplos/math.tex}}

\section{Estructuras complejas}
\begin{frame}[allowframebreaks]
  
\end{frame}

\section{Bibliograf�a}

\begin{frame}[allowframebreaks]
  \printbibliography\
\end{frame}

\end{document}
